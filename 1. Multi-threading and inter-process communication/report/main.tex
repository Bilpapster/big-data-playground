\documentclass[acmlarge]{acmart}


\AtBeginDocument{%
  \providecommand\BibTeX{{%
    Bib\TeX}}}
  \settopmatter{printacmref=false}

\renewcommand{\descriptionlabel}[1]{\hspace{\labelsep}\textit{#1}}
\usepackage{xcolor}
\newcommand{\todo}{{\color{red}\textbf{TODO} }}

% christos 1, 4
% vasilis 2, 3

\begin{document}

\title{Multi-Threading Programming and Inter-Process Communication}
\subtitle{M.Sc. course on ``Technologies for Big Data Analysis'' - Assignment 1}

\author{Christos Balaktsis (1234)}
\email{balaktsis@csd.auth.gr}
\author{Vasileios Papastergios (1234)}
\email{papster@csd.auth.gr}
\affiliation{
  \institution{Aristotle University}
  \city{Thessaloniki}
  \country{Greece}
}

\renewcommand{\shortauthors}{C. Balaktsis and V. Papastergios}
\maketitle

\section{Introduction}

The current document is a technical report for the first programming assignment in the M.Sc. course on \emph{Technologies for Big Data Analysis}, offered by the \emph{DWS M.Sc Program}\footnote{https://dws.csd.auth.gr/} of the Aristotle University of Thessaloniki, Greece. The course is taught by Professor Apostolos Papadopoulos~\footnote{https://datalab-old.csd.auth.gr/$\sim$apostol/}. The authors attended the course during their first year of Ph.D. studies at the Institution.

The assignment contains 4 sub-problems and is part of a series, comprising 3 programming assignments on the following topics:
\begin{description}
  \item[Assignment 1] Multi-threading Programming and Inter-Process Communication
  \item[Assignment 2] The Map-Reduce Programming Paradigm
  \item[Assignment 3] Big Data Analytics with Scala and Apache Spark
\end{description}
In this document we focus on Assignment 1 and its 4 sub-problems. We refer to them as \emph{problems} in the rest of the document for simplicity. The source code of our solution has been made available at \texttt{\small https://github.com/ Bilpapster/big-data-playground}.

\textbf{Roadmap}.
The rest of our work is structured as follows. We devote one section for each one of the 4 problems. That means problems 1, 2, 3 and 4 are presented in sections \ref{section:problem1}, \ref{section:problem2}, \ref{section:problem3} and \ref{section:problem4} respectively. For each problem, we first provide the problem statement, as given by the assignment. Next, we thoroughly present the reasoning and/or methodology we have adopted to approach the problem and devise a solution. Wherever applicable, we also provide insights about the source code implementation we have developed. For problems 2 and 4, we complete the respective sections with a discussion about alternatives or improvements the solution could accept, in order to successfully support more complex requirements. Finally, we conclude our work in section \ref{section:conclusion}.

\section{Problem 1: Concurrent Array-Vector Multiplication}
\label{section:problem1}
This is the section for the first problem.
\subsection{Problem Statement}

\section{Problem 2: Simulating a pandemic}
\label{section:problem2}
This is the section for the second problem.
\subsection{Problem Statement}

\section{Problem 3: Key-value server store}
\label{section:problem3}
This is the section for the third problem.
\subsection{Problem Statement}

\section{Problem 4: Multi-server producer-consumer interaction}
\label{section:problem4}
This is the section for the fourth problem.
\subsection{Problem Statement}

\section{Conclusion}
\label{section:conclusion}
This is the section for the conclusion.


\end{document}
\endinput
%%
%% End of file `sample-acmlarge.tex'.
